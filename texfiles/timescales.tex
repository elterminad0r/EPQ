\documentclass[a4paper,11pt]{article}
\title{Timescales}
\author{Izaak van Dongen}

% fonts
\usepackage[p,osf]{cochineal}
\usepackage[scale=.95,type1]{cabin}
\usepackage[cochineal,bigdelims,cmintegrals,vvarbb]{newtxmath}
% fixed width font with 80 chars per listing line
\usepackage[scaled=.94]{newtxtt}
\usepackage[cal=boondoxo]{mathalfa}

% pretty tables
\usepackage{booktabs}
\usepackage{longtable}

\usepackage[table]{xcolor}

% make the document take up more of the page
\usepackage[margin=1in,headheight=13.6pt]{geometry}

% no paragraph indent
\usepackage[parfill]{parskip}

% more advanced handling of utf8 and fonts or something. apparently good to have
\usepackage[utf8]{inputenc}
\usepackage[T1]{fontenc}

\begin{document}
    \maketitle

    As my project progressed, I sometimes tweaked my timescale to better reflect
    where the project was going. My timescale was never particularly detailed.

    \section{The original timescale}
    The very first timescale, from the first couple of weeks. Transcribed from
    the markdown file \texttt{timescale.md}

    \begin{center}
    {
    \renewcommand{\arraystretch}{1.3}
    \begin{longtable}{c p{0.7\textwidth}}
    \toprule
    Month & Vague Ideas \\
    \midrule
    December &
    Research string distance:
    \begin{enumerate}
        \item Hamming distance
        \item Levenshtein distance
        \item Find some sources
    \end{enumerate}\\
    January &
    Have written about string distances and how these apply to sequences in my
    dissertation \\
    February &
    Have implemented the code to find different distance metrics and apply this
    to a pool of codes (sieve) \\
    March &
    Have done more research on Hamming codes, and have independently generalised
    to base-4 \\
    April &
    Have implemented Hamming codes \\
    May &
    Do research on Hadamard codes, gray codes, and polynomial codes \\
    June &
    Have implemented these. \\
    \bottomrule
    \end{longtable}
    }
    \end{center}

    \section{Improvements}

    Developed over Christmas, as I had started my project and gotten a better
    feel of how it was going to happen. My original scope for the project had
    been a little ambitious. Again, transcribed from Markdown file.

    \begin{center}
    {
    \renewcommand{\arraystretch}{1.3}
    \begin{longtable}{c p{0.7\textwidth}}
    \toprule
    Month & More specific ideas \\
    \midrule
    \rowcolor{green} Completed &
    Start to gain proficiency with \LaTeX, and familiarise myself with the basic
    literature on error-correcting codes using resources like JSTOR. \\
    January &
    Do some research on Hamming codes, trying to find academic sources. Also
    have a look at how a Hamming code could apply to DNA (base 4). \\
    February &
    Start writing some code to implement Hamming encoding. Maybe also write some
    test plans/ unit tests. If time start writing about Hamming codes in
    dissertation.  \\
    March &
    Research string distance:
    \begin{enumerate}
        \item Hamming distance
        \item Levenshtein distance
        \item Find some sources
    \end{enumerate}
    Have written about string distances and how these apply to sequences in my
    dissertation \\
    April &
    Finish writing up Hamming codes in dissertation. \\
    May &
    Do research on Hadamard codes, and try to finish a basic implementation of
    some type of Hadamard code. \\
    June &
    Write up Hadamard codes. Maybe try to do some Hadamard visualisation with
    Postscript\ldots \\
    \bottomrule
    \end{longtable}
    }
    \end{center}

    \section{After exams}

    At this point I have a clearer idea of what has been done and what needs to
    happen. Also, at this point I brought the old timescales over to this
    document.

    It looks like I have crammed more into the last few months, but really it's
    just a higher level of detail.

    \begin{center}
    {
    \renewcommand{\arraystretch}{1.3}
    \begin{longtable}{c p{0.7\textwidth}}
    \toprule
    Month & Tasks \\
    \midrule
    \rowcolor{green} Completed &
    Start to gain proficiency with \LaTeX, and familiarise myself with the basic
    literature on error-correcting codes using resources like JSTOR.

    Do some research on Hamming codes, trying to find academic sources. Also
    have a look at how a Hamming code could apply to DNA (base 4).

    Start writing some code to implement Hamming encoding. Maybe also write some
    test plans/ unit tests. Writing about Hamming codes in
    dissertation.

    Have done some research on string distances:
    \begin{enumerate}
        \item Hamming distance
        \item Find some sources
    \end{enumerate}
    And written a short section in dissertation.

    Hadamard matrices research - various papers on types of Hadamard matrix and
    generators.

    Hadamard matrices generator - with the simple $2^n$ method I've managed to
    implement Hadamard matrices.

    Hadamard visualisation with Postscript! Some very pretty figures
    representing the matrices. \\

    May &
    Finish writing up Hamming codes in dissertation - last couple of bits
    regarding applications/limitations.

    If time, look at the Levenshtein distance.

    \\


    June &
    Also, start trying to write or at least prototype a Hadamard decoder
    (looking at generator matrices/linear algebra, or string distances). May
    need to fully implement string distance programs.

    See if I can implement Levenshtein distance, and if it will help with
    Hadamard decoder. \\

    July &
    Fully write up Hadamard matrices/codes, and the generator I'm using, with
    some recursive/iterative matrix notation.

    Make sure the dissertation is absolutely ready to go, double check all
    spelling, get people to proofread.
    If I have any more time I can look for some more sources for
    corroborration/extra citation. \\
    \bottomrule
    \end{longtable}
    }
    \end{center}

\end{document}
