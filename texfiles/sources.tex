\documentclass[a4paper,11pt]{article}
\title{Source tracker}
\author{Izaak van Dongen}

% fonts
\usepackage[p,osf]{cochineal}
\usepackage[scale=.95,type1]{cabin}
\usepackage[cochineal,bigdelims,cmintegrals,vvarbb]{newtxmath}
% fixed width font with 80 chars per listing line
\usepackage[scaled=.94]{newtxtt}
\usepackage[cal=boondoxo]{mathalfa}

% make the document take up more of the page
\usepackage[margin=0.3in,headheight=13.6pt]{geometry}

\usepackage{longtable}
\usepackage{booktabs}

\usepackage[parfill]{parskip}

\usepackage[utf8]{inputenc}
\usepackage[T1]{fontenc}

\usepackage[square,numbers]{natbib}

\usepackage[hidelinks]{hyperref}

\begin{document}
    \maketitle

    \section{Some general notes on my sources} \label{sourcenotes}

    Not all of my sources have been cross-referenced, although where possibly
    I've tried to demonstrate links between papers and their common subject
    matter. This doesn't really cause any problems though, as most of my
    sources are mathematical in nature. This means that really any source is
    either ``wrong'', or ``right'' about a theorem proposed by that source.
    This is pretty different from academic literature in more empirical fields,
    where cross-referencing is crucial.

    Overwhelmingly, peer-reviewed maths papers older than a couple of years will
    be correct, and my most important sources are all decades old. This binary
    nature of mathematics also means that most theorems won't really be produced
    in more than one paper, except in some cases where an alternative proof is
    offered.

    I would also argue that I have intrinsically been doing lots of research
    while writing my code. I have had to research what does and doesn't work,
    and write test plans to verify correct behaviour. I have also had to adapt
    what I've found in papers or online to work within the constraints of the
    programming language or my project. I have even written programs to simulate
    biological processes to verify the correct behaviour of my core programs.

    \section{Evaluation}

    \begin{center}
    {
    \renewcommand{\arraystretch}{2.0}
    \begin{longtable}{p{0.2\linewidth} p{0.35\linewidth} p{0.35\linewidth}} \toprule
    Source & Source content/usage & Source evaluation \\ \midrule

    Wikipedia, The Free Encyclopedia
    \cite{WikiCRISPR,WikiCas9,WikiHadamardCode,WikiHadamardMatrix,WikiHammingCode,
          WikiHamming74,WikiBarcoding} &

    I found several Wikipedia articles relating to my dissertation. I have
    decided to include them here as often I used them as a starting point for my
    research, eg to gain familiarity with the subject terminology and find
    convenient links to other Wikipedia articles, to get more of an idea of
    what's out there in the subject. The Wikipedia articles also provided a kind
    of baseline understanding of the material, which was then to be corroborated
    by my own research. They also provide their own links to other articles and
    sites which was useful in my research. &

    Obviously Wikipedia is not an academic source. In its defence, all of the
    information I found on there was then confirmed to be broadly correct by
    the actual literature of the subject (eg
    \cite{Codes1950Hamming,Communication1948Shannon}). However, on its, own,
    Wikipedia should not be considered a source, and these articles don't really
    underpin my dissertation. In the end, I decided still to cite them as they
    were a part of my research, as stated previously. \\

    Error detecting and error correcting codes \cite{Codes1950Hamming} &

    Gives a construction of Hamming codes and lays the foundation for many of
    the concepts around channel-coding that I use in my dissertation. &

    This, is a paper by R. W. Hamming, who contributed much to modern
    error-correcting codes. It was published over 50 years ago in a respected
    paper by a respected author and has not only stood the test of time but
    also formed much of the basis of modern communication theory and can
    certainly be trusted.  \\

    A mathematical theory of communication \cite{Communication1948Shannon} &

    This is also a useful paper for establishing the theory of communicative
    coding. It is useful for several definitions and general limits. &

    This paper, together with \cite{Codes1950Hamming} are generally
    considered to be the seminal works on coding theory. This lays much of the
    groundwork for communication theory and gives a more general definition of
    the Hamming Code. Similarly to \cite{Codes1950Hamming}, this is a paper
    that can be trusted \\

    Generalized dna barcode design based on hamming codes \cite{HammingBarcodes2012BystrykhLeonid} &

    This article is very much relevant to what my project is about. It doesn't
    seem to be very clear though, and it uses a seemingly non-optimal form of
    parity. However it provides helpful insight into what actual researchers in
    the field are doing and have done with these ideas. &

    It is an article in a reputable, peer reviewed journal by researchers in
    the field so it can probably be trusted. The journal is slightly less
    well-known than some others I've made use of. \\

    Introduction to coding theory \cite{CodeIntro2010Guruswami} &

    This is more of a crash-course in coding theory, which was useful in
    developing a basic understanding but can be built on. &

    It's not an academic article so it hasn't been peer reviewed and wasn't
    written with the aim of academic rigour. However it was produced by a
    mathematician at a prestigious university so is likely to have some merit.
    It's also specifically aimed at the teaching of the subject rather than
    academic description which may be useful for my purposes. \\

    Polynomial codes: an optimal design for high-dimensional coded matrix multiplication \cite{PolynomialCodes2017MaddahAvestimehr} &

    This source provides an explanation of a facet of the polynomial code. This
    code type didn't end up being used for my project but the source was useful
    in explaining why not. &

    This source is very technically detailed, which isn't necessarily bad, but
    this source feels very dense and can very well be difficult to understand
    for anyone not familiar with this specific subfield of coding theory.  Of
    course it does have academic merit as it's a published, peer reviewed
    article. \\

    Families of Hadamard z2z4q8-codes \cite{HadamardZ2Z2012RioRifa} &

    This is an article relating to a specific subset of Hadamard codes. I
    didn't end up using these codes but this source provided some broader
    context for what Hadamard codes are/can be used for. &

    This is again a highly technical source but is generally reputable aside
    from that. \\

    Hadamard matrices and their applications \cite{HadamardMatrices1978HedayatWallis} &

    This is a reasonably low-level paper about Hadamard codes. This was useful
    for my dissertation as I didn't use any overly complicated Hadamard
    constructions, and this was a useful source to cite, while also being
    trustworthy\ldots &

    It's an academic paper from a trustworthy journal that has stood the test
    of time, and aside from that doesn't really make any outrageous claims,
    forming more of a useful and approachable summary of the subject matter. \\

    The search for Hadamard matrices \cite{HadamardSearch1963GolombBaumert} &

    This paper gives a useful initial summary of the `basic' construction of
    Hadamard matrices (of size $2^n \times 2^n$)  which I use, corroborating
    \cite{HadamardMatrices1978HedayatWallis}. It also provides a nice summary
    of other constructions, which I used to strengthen my case for not using
    them as they seemed unnecessary. &

    It's a trustworthy article, and again provides more of a summary. It's
    written in a relatively friendly manner. \\

    Hadamard matrices and their designs: A coding-theoretic approach \cite{HadamardCodingTheoretic1992AssmusKey} &

    This paper gives a number of very mathematically involved constructions of
    Hadamard matrices. I didn't end up using any of these but it provided a
    useful further background around Hadamard constructions, and could be
    compared/contrasted with \cite{HadamardSearch1963GolombBaumert} &

    It's a trustworthy paper from a trustworthy journal and trustworthy
    authors. \\

    Hadamard designs \cite{HadamardDesigns1972Spence} &

    This paper is about Hadamard designs for alphabets of size $n$ where $n
    \neq 2$. This is potentially very interesting with respect to my project as
    the DNA alphabet size $n$ is 4. This source was used to provide citations in my
    general discussion about conversion between different alphabets. &

    It's a trustworthy paper although it is on an apparently quite obscure
    subject, making it quite a rare occurrence. This doesn't really reflect on
    the trustworthiness of the source due to the reasons described in section
    \ref{sourcenotes}.  \\

    Lifted polynomials over ${F}_{16}$ and their applications to dna codes \cite{PolynomialDNA2013OztasSiap} &

    This is a paper about a highly specific and even more advanced class of
    codes. It involved a lot of field theory, so for someone with an A-level
    knowledge of maths, will require a lot of further reading. However it does
    directly talk about applications of codes to DNA, which is useful to
    validate that this is a field with some merit. &

    It's a very reputable and technically accurate paper, but again is
    incredibly dense and required much more knowledge than I have, so I've not
    been able to effectively leverage it as a citation.

    \\

    Codes, not ciphers \cite{CodesCiphers1010Baylis} &

    This is an article providing description and intruction of coding theory,
    and contrasting that with cryptography. It is aimed at a `school \ldots
    level'. This means that it was useful to gain a quick overview of what
    coding theory is, but didn't really help with any major lifting. &

    It's from a very reputable paper, and is also easily accessible. This makes
    it a good, reliable resource for introduction to the topic.

    \\

    Error correcting codes: Practical origins and mathematical implications \cite{CodesOrigins1978Pless} &

    This is a very short little paper, which provides a short history of the
    concept of codes. This makes it quite a special little paper, and quite nice
    to have read to gain a better contextual understanding of the field.
    However, again, there's not much to cite. &

    Again, from a highly trustworthy paper and an established mathematician.
    Occasionally it uses some pretty advanced mathematical analogies, but is
    certainly a very solid source.

    \\

    Boole and the algebra of logic \cite{BooleRecords1956Kneale} &

    This is used as a small citation on Boolean algebra, which I needed when I
    was writing about my decision to use boolean inversion rather than additive
    inversion for my Hadamard construction. This isn't a very mathematically
    advanced source, but this is in line with my project as Boolean algebra was
    only a small part of a subset of my project, and the main focus was not on
    Boolean algebra. &

    This isn't really an article but a set of notes as it was published in
    `Notes and Records of the Royal Society of London'. While this means it's
    not the epitome of academic rigour, it's still a useful and factual source
    on Boolean algebra from the Royal Society.

    \\

    The degeneracy of the genetic code and Hadamard matrices.  \cite{DegeneracyHadamard2008Petoukhov} &

    This is a great paper talking about how Hadamard matrices can be applied to
    and modelled against genetic code. The really cool thing is that this is a
    paper from a biomechanical perspective, rather than a mathematician's. This
    highlights the truly awesome thing about working at an intersection of
    fields. Aside from that, used for some backing up of the application of
    Hadamard matrices. &

    This paper is a little wonky in terms of academic merit - it's not published
    in any well known journal, which might make it seem somewhat questionable.
    However, it is mathematical in nature, and is mathematically sound. This
    paper would have nothing to gain from misleading anyone anywa.

    \\

    Construction of multilevel Hadamard matrices with small alphabet \cite{MultilevelConstruction2008TrinhFan} &

    This is a very interesting topic, as it potentially deals with the major
    limitation of Hadamard matrices - them having an alphabet of 2. Being able
    to construct a Hadamard matrix with different alphabet (ideally 4) would be
    incredibly useful. However, this paper doesn't quite go so far as to
    explicitly guide how that could be possible, so it's just another citation
    for theoretical extensions to this project. &

    It is a perfectly sound paper. Only quibble is that it's a little hard to
    gain access to.

    \\

    Decoding the hamming code \cite{DecodingHamming2006Eherenborg} &

    This is a tiny article offering a quick introduction to coding theory and
    Hamming codes. It contextualises the Hamming code in an interesting new way
    that I quite like and helped me really click with what they could be used
    for and what they were, so I used this source as an excellent other take on
    Hamming codes. &

    Again, despite not being very academic in and of itself it is from a very
    prestigious university and organisation, and can't really be faulted in any
    way.

    \\

    PEP 257: Docstring conventions \cite{PEPDocstrings2014Goodger} &

    This is part of the Python style guide, used to back up some of my
    assertions about coding practices in my Definitions section. &

    Of course, it isn't remotely academic and verges on being an appeal to
    authority, but it is a well respected style guide with some explicit
    reasoning behing the decisions being laid out.

    \\

    WHAT IS CRISPR-CAS9? \cite{CRISPRDailymail} &

    This is a short ``article'' offering a brief introduction of the principles
    of CRISPR-CAS9. The best part is probably a diagram produced by MIT. It's
    not particularly informative aside from the most general, basic sense. &

    It's literally not from a scientific publication and has no academic merit.
    This is a bad source and does not prove or reinforce anything.

    \\

    American Standard Code for Information Interchange \cite{ASCII1963ASA} &

    This is the standard for ASCII, which I briefly mention while discussing
    encodings, specifically different ways to encode the letter `A'. &

    This is \textbf{the} authoritative source for this purpose, as it is
    literally the standard that I am talking about.  \\

    Thoughts on inverse orthogonal matrices, simultaneous
    signsuccessions, and \ldots \cite{OrthogonalMatrix1867Sylvester} &

    This is the paper by Sylester that introduces his construction of the Walsh
    matrices. This is important to my project as this is the construction I
    mainly focus on/have implemented. &

    Again, academic paper with significant clout, and mathematically, it works,
    so it's good enough for me. \\

    CRISPR/Cas, the Immune System of Bacteria and Archaea
    \cite{CRISPRImmune2010Horvath} &

    An article discussing the usage of CRIPR/Cas9 to various biological things.
    This helps to contextualise the potential applications of my outcome in the
    greater scheme of things. &

    Once again, a fine paper with not much to worry about. \\

    Responding to {CRISPER}\-/\-Cas9 [sic] \cite{CRISPER2016Finneran} &

    This is short article discussing the ramifications/applications and ethics
    behind CRISPR. Again, used in the discussion of why CRISPR is interesting. &

    Note that the author misspelled ``CRISPR'' in the actual title - while this
    isn't a disqualifying blow, it's a bit iffy. However, some valid points are
    made. \\

    RNA-directed gene editing specifically eradicates latent and prevents new
    HIV-1 infection \cite{RNAHIV2014Hu} &

    Article that is also used to provide some more info about how the bigger
    picutre of gene editing can do good in the world. &

    Good academic scholarly article etc.  \\

    Perfect Parity Patterns: 11243 \cite{Parity2008Knuth} &

    This is an article by the famous Donald Knuth involving parity, which I use
    as a citation for the definition of parity. &

    It was produced by an absolute icon and again is an academic, peer reviewed
    article so is certainly trustworthy.  \\

    \bottomrule
    \end{longtable}
    }
    \end{center}

\bibliographystyle{agsm}
\bibliography{sources}
\end{document}
