\documentclass{article}
\title{Source tracker}
\author{Izaak van Dongen}

\usepackage{fullpage}

\usepackage{amsmath}
\usepackage{amsfonts}
\usepackage{commath}

\usepackage{longtable}
\usepackage{booktabs}

\usepackage[parfill]{parskip}

\usepackage[utf8]{inputenc}
\usepackage[T1]{fontenc}

\usepackage[square]{natbib}

\usepackage{graphicx}

\graphicspath{ {images/} }

\usepackage{listings}
\usepackage{color}

\definecolor{codegreen}{rgb}{ 0,0.6,0}
\definecolor{codegray}{rgb}{0.5,0.5,0.5}
\definecolor{codepurple}{rgb}{0.58,0,0.82}
\definecolor{backcolour}{rgb}{0.95,0.95,0.92}
\lstdefinestyle{mystyle}{
    backgroundcolor=\color{backcolour},
    commentstyle=\color{codegreen},
    keywordstyle=\color{magenta},
    numberstyle=\tiny\color{codegray},
    stringstyle=\color{codepurple},
    basicstyle=\footnotesize,
    breaklines=true,
    captionpos=b,
    keepspaces=true,
    numbers=left,
    numbersep=5pt,
    showspaces=false,
    showstringspaces=false,
    showtabs=false,
    tabsize=2
}

\lstset{style=mystyle}

\lstset{
    literate={~} {$\sim$}{1}
}

\begin{document}
    \maketitle

    \section{Some general notes on my sources} \label{sourcenotes}

    Not all of my sources have been cross-referenced, although where possibly
    I've tried to demonstrate links between papers and their common subject
    matter. This doesn't really cause any problems though, as most of my
    sources are mathematical in nature. This means that really any source is
    either ``wrong'', or ``right'' about a theorem proposed by that source.
    This is pretty different from academic literature in more empirical fields,
    where cross-referencing is crucial.  Overwhelmingly, peer-reviewed maths
    papers older than a couple of years will be correct, and my most important
    sources are all decades old. This binary nature of mathematics also means
    that most theorems won't really be produced in more than one paper, except
    in some cases where an alternative proof is offered.

    \section{Evaluation}

    \begin{center}
    {
    \renewcommand{\arraystretch}{2.0}
    \begin{longtable}{p{0.2\linewidth} p{0.35\linewidth} p{0.35\linewidth}} \toprule
    Source & Source content/usage & Source evaluation \\ \midrule

    Error detecting and error correcting codes \citep*{Codes1950Hamming} &

    Gives a construction of Hamming codes and lays the foundation for many of
    the concepts around channel-coding that I use in my dissertation. &

    This, is a paper by R. W. Hamming, who contributed much to modern
    error-correcting codes. It was published over 50 years ago in a respected
    paper by a respected author and has not only stood the test of time but
    also formed much of the basis of modern communication theory and can
    certainly be trusted.  \\

    A mathematical theory of communication \citep*{Communication1948Shannon} & 

    This is also a useful paper for establishing the theory of communicative
    coding. It is useful for several definitions and general limits. &

    This paper, together with \citep*{Codes1950Hamming} are generally
    considered to be the seminal works on coding theory. This lays much of the
    groundwork for communication theory and gives a more general definition of
    the Hamming Code. Similarly to \citep*{Codes1950Hamming}, this is a paper
    that can be trusted \\

    Generalized dna barcode design based on hamming codes \citep*{HammingBarcodes2012BystrykhLeonid} &

    This article is very much relevant to what my project is about. It doesn't
    seem to be very clear though, and it uses a seemingly non-optimal form of
    parity. However it provides helpful insight into what actual researchers in
    the field are doing and have done with these ideas. &

    It is an article in a reputable, peer reviewed journal by researchers in
    the field so it can probably be trusted. The journal is slightly less
    well-known than some others I've made use of. \\

    Introduction to coding theory \citep*{CodeIntro2010Guruswami} &

    This is more of a crash-course in coding theory, which was useful in
    developing a basic understanding but can be built on. &

    It's not an academic article so it hasn't been peer reviewed and wasn't
    written with the aim of academic rigour. However it was produced by a
    mathematician at a prestigious university so is likely to have some merit.
    It's also specifically aimed at the teaching of the subject rather than
    academic description which may be useful for my purposes. \\

    Polynomial codes: an optimal design for high-dimensional coded matrix multiplication \citep*{PolynomialCodes2017MaddahAvestimehr} &

    This source provides an explanation of a facet of the polynomial code. This
    code type didn't end up being used for my project but the source was useful
    in explaining why not. &

    This source is very technically detailed, which isn't necessarily bad, but
    this source feels very dense and can very well be difficult to understand
    for anyone not familiar with this specific subfield of coding theory.  Of
    course it does have academic merit as it's a published, peer reviewed
    article. \\

    Families of Hadamard z2z4q8-codes \citep*{HadamardZ2Z2012RioRifa} & 

    This is an article relating to a specific subset of Hadamard codes. I
    didn't end up using these codes but this source provided some broader
    context for what Hadamard codes are/can be used for. &

    This is again a highly technical source but is generally reputable aside
    from that. \\

    Hadamard matrices and their applications \citep*{HadamardMatrices1978HedayatWallis} &

    This is a reasonably low-level paper about Hadamard codes. This was useful
    for my dissertation as I didn't use any overly complicated Hadamard
    constructions, and this was a useful source to cite, while also being
    trustworthy\ldots &

    It's an academic paper from a trustworthy journal that has stood the test
    of time, and aside from that doesn't really make any outrageous claims,
    forming more of a useful and approachable summary of the subject matter. \\

    The search for Hadamard matrices \citep*{HadamardSearch1963GolombBaumert} &

    This paper gives a useful initial summary of the `basic' construction of
    Hadamard matrices (of size $2^n \times 2^n$)  which I use, corroborating
    \citep*{HadamardMatrices1978HedayatWallis}. It also provides a nice summary
    of other constructions, which I used to strengthen my case for not using
    them as they seemed unnecessary. &

    It's a trustworthy article, and again provides more of a summary. It's
    written in a relatively friendly manner. \\

    Hadamard matrices and their designs: A coding-theoretic approach \citep*{HadamardCodingTheoretic1992AssmusKey} & 

    This paper gives a number of very mathematically involved constructions of
    Hadamard matrices. I didn't end up using any of these but it provided a
    useful further background around Hadamard constructions, and could be
    compared/contrasted with \citep*{HadamardSearch1963GolombBaumert} & 
    
    It's a trustworthy paper from a trustworthy journal and trustworthy
    authors. \\

    Hadamard designs \citep*{HadamardDesigns1972Spence} & 

    This paper is about Hadamard designs for alphabets of size $n$ where $n
    \neq 2$. This is potentially very interesting with respect to my project as
    the DNA alphabet size $n$ is 4. This source was used to provide citations in my
    general discussion about conversion between different alphabets. &

    It's a trustworthy paper although it is on an apparently quite obscure
    subject, making it quite a rare occurrence. This doesn't really reflect on
    the trustworthiness of the source due to the reasons described in section
    \ref{sourcenotes}.  \\

    Lifted polynomials over ${F}_{16}$ and their applications to dna codes \citep*{PolynomialDNA2013OztasSiap} & 

    \\

    Codes, not ciphers \citep*{CodesCiphers1010Baylis} &

    \\

    Error correcting codes: Practical origins and mathematical implications \citep*{CodesOrigins1978Pless} &

    \\

    Boole and the algebra of logic \citep*{BooleRecords1956Kneale} &

    This is used as a small citation on Boolean algebra, which I needed when I
    was writing about my decision to use boolean inversion rather than additive
    inversion for my Hadamard construction. This isn't a very mathematically
    advanced source, but this is in line with my project as Boolean algebra was
    only a small part of a subset of my project, and the main focus was not on
    Boolean algebra. &

    This isn't really an article but a set of notes as it was published in
    `Notes and Records of the Royal Society of London'. While this means it's
    not the epitome of academic rigour, it's still a useful and factual source
    on Boolean algebra from the Royal Society.

    \\

    The degeneracy of the genetic code and Hadamard matrices. \citep*{DegeneracyHadamard2008Petoukhov}

    \\

    Construction of multilevel Hadamard matrices with small alphabet \citep*{MultilevelConstruction2008TrinhFan}

    \\

    Decoding the hamming code \citep*{DecodingHamming2006Eherenborg}

    \\

    \bottomrule
    \label{tab:sourceeval}
    \end{longtable}
    }

    \ref{tab:sourceeval}
    \end{center}

    \bibliographystyle{agsm}
    \bibliography{sources}

\end{document}
