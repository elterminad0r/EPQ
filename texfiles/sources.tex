\documentclass{article}
\title{Source tracker}
\author{Izaak van Dongen}

\usepackage{fullpage}

\usepackage{amsmath}
\usepackage{amsfonts}
\usepackage{commath}

\usepackage{longtable}
\usepackage{booktabs}

\usepackage[parfill]{parskip}

\usepackage[utf8]{inputenc}
\usepackage[T1]{fontenc}

\usepackage[square]{natbib}

\usepackage{graphicx}

\graphicspath{ {images/} }

\usepackage{listings}
\usepackage{color}

\definecolor{codegreen}{rgb}{ 0,0.6,0}
\definecolor{codegray}{rgb}{0.5,0.5,0.5}
\definecolor{codepurple}{rgb}{0.58,0,0.82}
\definecolor{backcolour}{rgb}{0.95,0.95,0.92}
\lstdefinestyle{mystyle}{
    backgroundcolor=\color{backcolour},
    commentstyle=\color{codegreen},
    keywordstyle=\color{magenta},
    numberstyle=\tiny\color{codegray},
    stringstyle=\color{codepurple},
    basicstyle=\footnotesize,
    breaklines=true,
    captionpos=b,
    keepspaces=true,
    numbers=left,
    numbersep=5pt,
    showspaces=false,
    showstringspaces=false,
    showtabs=false,
    tabsize=2
}

\lstset{style=mystyle}

\lstset{
    literate={~} {$\sim$}{1}
}

\begin{document}
    \maketitle

    \begin{center}
    {
    \renewcommand{\arraystretch}{2.0}
    \begin{longtable}{p{0.2\linewidth} p{0.35\linewidth} p{0.35\linewidth}} \toprule
    Source & Source content & Source evaluation \\ \midrule

    Error detecting and error correcting codes \citep*{Codes1950Hamming} &

    This is a paper by R. W. Hamming, who contributed much to modern
    error-correcting codes. One of the main encoding types used in my project
    is even named the Hamming code. It has formed much of the basis of modern
    communication theory and can certainly be trusted. &
    
    daadsf ads fads fadsf asdf ads fasdf 
    daadsf ads fads fadsf asdf ads fasdf 
    daadsf ads fads fadsf asdf ads fasdf 
    daadsf ads fads fadsf asdf ads fasdf 
    
    \\

    A mathematical theory of communication \citep*{Communication1948Shannon} & 

    This paper, together with \citep*{Codes1950Hamming} are generally considered
    to be the seminal works on coding theory. This lays much of the groundwork
    for communication theory and gives a more general definition of the Hamming
    Code. \\

    Generalized dna barcode design based on hamming codes \citep*{HammingBarcodes2012BystrykhLeonid} &

    This article is very much relevant to what my project is about. It doesn't
    seem to be very clear though, and it uses a seemingly inoptimal form of
    parity. However it provides helpful insight into what actual researchers in
    the field are doing and have done with these ideas. \\

    Introduction to coding theory \citep*{CodeIntro2010Guruswami} &

    This is not a very good academic source but gives a good informal overview
    of coding theory. \\

    Polynomial codes: an optimal design for high-dimensional coded matrix multiplication \citep*{PolynomialCodes2017MaddahAvestimehr} &

    This source is very technically detailed, which isn't necessarily a bad
    thing but makes it pretty dense. Potentially very useful for a complicated
    understanding of polynomial codes, although I'm not sure if I'll use
    polynomial codes. \\

    Families of hadamard z2z4q8-codes \citep*{HadamardZ2Z2012RioRifa} & 

    This source turned out not to be very useful as it only relates to a highly
    specific class of Hadamard code. Other simpler tutorials on the internet
    are much more useful. \\

    Hadamard matrices and their applications \citep*{HadamardMatrices1978HedayatWallis} &

    This is a far more appropriate paper that gives a more general overview of
    what a hadamard matrix is and can be used for. Very useful overall. \\

    The search for hadamard matrices \citep*{HadamardSearch1963GolombBaumert} &

    This paper gives a very good overview of Hadamard's original construction
    of the $2^n \times 2^n$ matrices. \\

    Hadamard matrices and their designs: A coding-theoretic approach \citep*{HadamardCodingTheoretic1992AssmusKey} & 

    This paper gives a number of very mathematically involved constructions of
    Hadamard matrices. I didn't end up using any of these but it provided a
    useful further background around Hadamard constructions. \\

    Hadamard designs \citep*{HadamardDesigns1972Spence} & 

    This paper is about Hadamard designs for alphabet sizes of $n$ where $n
    \neq 2$. This is potentially very useful information as I am concerned with
    DNA barcodes, ie $n = 4$. \\

    Lifted polynomials over ${F}_{16}$ and their applications to dna codes \citep*{PolynomialDNA2013OztasSiap} & 

    \\

    Codes, not ciphers \citep*{CodesCiphers1010Baylis} &

    \\

    Error correcting codes: Practical origins and mathematical implications \citep*{CodesOrigins1978Pless} &

    \\

    Boole and the algebra of logic \citep*{BooleRecords1956Kneale} &

    This isn't a very mathematically advanced source but my project doesn't
    need to be very highly complex in this area. This is used for a quick
    citation on how I can adapt Hadamard generation with additive negation to
    the Boolean (algebra) system used by computers.

    \\

    The degeneracy of the genetic code and hadamard matrices. \citep*{DegeneracyHadamard2008Petoukhov}

    \\

    Construction of multilevel hadamard matrices with small alphabet \citep*{MultilevelConstruction2008TrinhFan}

    \\

    Decoding the hamming code \citep*{DecodingHamming2006Eherenborg}

    \\

    \bottomrule
    \label{tab:sourceeval}
    \end{longtable}
    }

    \ref{tab:sourceeval}
    \end{center}

    \bibliographystyle{agsm}
    \bibliography{sources}

\end{document}
