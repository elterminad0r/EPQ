\documentclass{article}
\title{EP diary}
\author{Izaak van Dongen}

% fonts
\usepackage[p,osf]{cochineal}
\usepackage[scale=.95,type1]{cabin}
\usepackage[cochineal,bigdelims,cmintegrals,vvarbb]{newtxmath}
% fixed width font with 80 chars per listing line
\usepackage[scaled=.94]{newtxtt}
\usepackage[cal=boondoxo]{mathalfa}

% make the document take up more of the page
\usepackage[margin=0.3in,headheight=13.6pt]{geometry}

\usepackage{longtable}
\usepackage{booktabs}

\usepackage[parfill]{parskip}

\usepackage[utf8]{inputenc}
\usepackage[T1]{fontenc}

\begin{document}
    \maketitle

    \begin{center}
    {
    \renewcommand{\arraystretch}{2.0}
    \begin{longtable}{p{0.1\linewidth} p{0.2\linewidth} p{0.2\linewidth} p{0.2\linewidth} p{0.2\linewidth}} \toprule

    Date & Work covered & Problems encountered & What next? & Notes if applicable \\ \midrule

    2017-11-16 &

    Filled out PPR &

    Wasn't yet entirely sure about fine details &

    Complete the timescale &

    (entered retroactively) \\ 2017-11-23 &

    Decided in what format to store this diary and then created it &

    Also considered tsv and what date format to use &

    Improve the timeline and add to the diary &

    \\ 2017-11-23 &

    Wrote a short script to display the diary in HTML form &

    Making this convenient and debugging/considering libraries &

    Add to the diary and maybe do something similar for timescale &

    \\ 2017-12-04 &

    Worked through the material on researching and referencing (AO2) from the
    sharepoint (to catch up on missed lesson) &

    Not sure what the abbreviations should be used for &

    Start to apply this to my own research &

    (This was entered on 12-07) \\ 2017-12-07 &

    Evaluated some previous pupils' EPs - found out that lots of marks are
    awarded for things other than actual writing &

    How the marking actually works &

    Expand my own timescale/diary and PPR &

    \\ 2017-12-14 &

    Got my dad to peer review my project so far &

    The timescale and diary can be a little unclear and haven't started
    dissertation &

    Try to add more information to each diary entry and perhaps develop a Gantt
    chart &

    \\ 2017-12-20 &

    Did some research on Hamming codes and took some notes on the
    application/implementation &

    0-based vs 1-based indexing &

    Start to translate to actual code &

    \\ 2017-12-30 &

    Started planning how to implement a Hamming encoder in binary Python &

    &

    Start to implement the binary data formats and algorithms &

    \\ 2017-12-30 &

    I also edited my timescale, to reflect what I think is a more realistic view
    of what will happen and when it will happen. &

    It can be tricky to estimate how long tasks will take, and I think I
    especially get easily distracted if I'm working on something I find
    interesting. &

    Probably write some more code.

    \\ 2018-1-4 &

    Started to implement the Hamming encoder &

    Indexing into data vs code &

    Finish the program and write test plan &

    Uses list/bit representation \\ 2018-1-4 &

    Considered available primary research &

    Survey is very much inapplicable here &

    Contact Leo and Aylwyn &

    \\ 2018-1-7 &

    Completed and implemented test plan for Hamming encoder &

    Remembering / researching how to use unittest &

    Eventually add some more test cases/use real world or generated data &

    Start to work more on dissertation/how to present program \\ 2018-1-11 &

    Decided on using LaTeX to write my dissertation and researched and
    experimented with different themes &

    Latex workflow (using pdflatex and repeated compilation) and some latex
    bugs &

    Start to write the document and look at bibtex &

    \\ 2018-1-22 &

    Started to develop the LaTeX document structure and made a rough first
    draft of dissertation (testing ideas and aesthetics) &

    What font size to use &

    Also consider how diagrams might be incorporated &

    I can reuse some of the things found doing computing assignments \\
    2018-1-25 &

    Reviewed progress so far &

    OneDrive didn't really sufficiently reflect what I'd done &

    Fill in bits of diary and upload stuff to OneDrive &

    \\ 2018-1-25 &

    Started looking at TikZ to present a mindmap of my project/initial thoughts
    &

    Using tikz without graphical bugs and realised I wasn't entirely sure how I
    would present my project as a mindmap &

    Continue to work on mindmap/look at how to include it in dissertation &

    \\ 2018-1-29 &

    Researched BibTex and set up the .bib file w/ makefile. Got two original
    sources (see sources.bib) &

    How to cite an introductory-type PDF in bibtex (ended up using @misc) &

    Continue to research and add sources. Also add a source tracker rather than
    just a bibliography &

    \\ 2018-2-1 &

    Found two more sources and cited them in .bib file &

    Choosing appropriate tags and figuring out how to cite with bibtex &

    Continue to find sources/think about where the sources can be incorporated
    in dissertation &

    \\ 2018-2-1 &

    Retroactively documented problems and approach in diary &

    Frustration with inadvertently incorrect data entry &

    Continue to fully document what I do &

    \\ 2018-2-1 &

    Updated the Python HTML compiler script to warn of bad csv formatting (to
    prevent frustration) &

    &

    I really want to improve the format and content of my timescale &

    \\ 2018-2-13 &

    Found paper on polynomial codes which I took some notes on and then added
    to my bibliography &

    It seems quite sophisticated and is a little dense &

    At some point consider how these might be implemented &

    \\ 2018-2-16 &

    Found a good paper on Hadamard codes and did some other research on
    Hadamard codes &

    The actual paper doesn't seem entirely applicable &

    Implement basic Hadamard codes and isomorph to DNA &

    \\ 2018-2-19 &

    Started to write the Python code to generate Hadamard matrices &

    Whether to use numpy or make a custom matrix class &

    Write a test suite and complete the code &

    \\ 2018-3-1 &

    Finished the Hadamard matrix program &

    Decided not to use an object oriented model due to the introduced overhead
    and instead used primitive lists. I also decided to modify the 1/-1 system
    to be boolean 1/0 as this fits my requirements better &

    Finish the testing scheme &

    I also found another more appropriate source to use for hadamard codes \\
    2018-3-7 &

    Signed up for a JSTOR account under the school network after finding out
    that could be done on the library sharepoint &

    &

    Use this to find more papers and see if any existing sources I have are on
    there so I can use JSTOR to cite them &

    \\ 2018-3-7 &

    Found a number of papers on Hadamard codes and polynomial codes &

    Some of them contain some very dense mathematics in niche fields &

    I want to find more sources and also I want to figure out how to use a
    Harvard style with BibTeX &

    \\ 2018-3-8 &

    I spent a while configuring natbib for use with my dissertation so now my
    references and citations are all in the Harvard style &

    There was a conflict between natbib and savetrees with the extreme
    parameter so I had to change to fullpage &

    Write a Hadamard tester and start to write about Hadamard codes/matrices in
    dissertation &

    \\ 2018-3-9 &

    Changed my source tracker so that I now have separate columns for
    evaluation and content to provide clearer evidence of research &

    I spent about an hour trying to make my LaTeX table pretty (worth it) &

    Start writing the actual evaluations/content summaries for each source &

    \\ 2018-3-11 &

    I wrote the evaluations for a number of sources. &

    As my sources are currently almost all articles/papers, the evaluation
    becomes a little monotonous. This also perhaps doesn't fully demonstrate
    that I can also evaluate less reputable sources. &

    Find a ``bad'' source (eg Daily Mail article or a forum post) to add some
    variation. &

    \\ 2018-3-15 &

    Today I figured out how to use latexmk instead of just pdflatex or latex.
    This made the work of compiling a lot easier. &

    I had to install latexmk and create an appropriate config file, and figure
    out how to use the various parameters. &

    Incorporate this into my workflow, and start setting out my "problems etc"
    document more thoroughly. &

    \\ 2018-3-18 &

    I spent some time writing about problems I faced and how I overcame them
    with more detail, using my diary. &

    It was a bit difficult to succinctly phrase how everything was ``overcome''
    &

    I should fill this in more fully &

    \\ 2018-3-22 &

    We did the peer review in class, where we each assessed each other's work
    wrt AOs. &

    It was a bit of a challenge to immerse yourself in someone else's project,
    and also difficult to judge how to apply AOs (not too harshly, not too
    nicely). The feedback I received was a little lacklustre. &

    I should take note of the feedback I received and respond to it/ work around
    it.

    \\ 2018-3-25 &

    I spent some time writing my dissertation, focusing on the explanation
    and visualisation of the Hamming distance, and parity. &

    I spent some time thinking about good ways to communicate visually and in
    words the principles behind the Hamming distance. &

    It would be good to produce some graphics to illustrate parity

    \\ 2018-3-29 &

    I worked on explaining the idea of parity and produced a sample row/column
    parity encoding &

    I had some trouble with including the graphic I had made in my dissertation,
    but eventually I figured it out. &

    While of course relaxing over half term, I think I should also do some more
    programming.

    \\ 2018-4-8 &

    After installing my new operating system over half term, I successfully
    moved my project over to it. This was made very easy due to my usage of git
    as a versioning system. I also had to install the whole \LaTeX toolchain,
    but this went off without a hitch.

    \\ 2018-4-8 &

    Having succesfully migrated, I started work on the Hamming decoder &

    I had several bugs to do with indexing - I got a lot of off-by-one errors. I
    also haven't fully tested it yet &

    I should write some unit tests &

    \\ 2018-4-12 &

    I re-read the specification for an EPQ and saw that I needed to include some
    student information (candidate no., etc), and then added this. &

    It was easy enough to put it, but I wanted all the information to line up
    nicely as well &

    I should probably write some more actual bulk of the dissertation 

    \\ 2018-4-19 &

    I wrote a bit about complexity and asymptotic complexity, and produced a
    graph of $\sqrt{x}$ vs $\log_2 x$ &

    Graphing things in \LaTeX is apparently very complicated. I ended up
    producing something worthwhile with pgfplots. &

    I should get a move on with quaternary Hamming stuff.

    \\ 2018-4-26 &

    After finding a bug in the Hamming implementation, I spent some time
    debugging and writing more tests. &

    This bug was quite tiresome to find due to the size of my code base. &

    I should definitely be writing more unit tests.

    \\ 2018-5-4 &

    I produced a graphic in Postscript to show how Hamming parity coverage
    works. &

    Again had some trouble but eventually figure out how to use includegraphic
    together with a figure and graphicx. &

    I'm starting to run out of code I've written to write about, so definitely
    more programming after exams.

    \\ 2018-5-28 &

    I have totally changed my preamble, so the entire look and feel is much
    different, and it's less instantly recognizable as a basic \LaTeX\ article. &

    I had some trouble with font installation and location, but got there in the
    end. &

    I could write something about how this shows my progress in skill with
    \LaTeX. Also get cracking with coding.

    \\ 2018-5-29 &

    I finished off the Hamming decoder, adding compatibility with generic bases
    and writing tests for it, and wrote some of the corresponding section of the
    dissertation &

    There were several nasty bugs with defaults and indices, and just an overall
    bad approach to parsing arguments. However, as I have pulled through my
    whole code base should be more maintainable now. &

    I should also have a go at writing direct conversion between binary and
    quaternary, and maybe some helper mutation/verification scripts.

    \\ 2018-5-30 &

    I wrote two programs: one simulating mutation in input data, and one
    verifying the integrity of input data. This was very useful as it lets me
    "trial" my program without wasting any bioinformaticians' money. &

    I had to think a bit about the design of the mutation simulator, as it can
    be quite intricate to simulate a biological process in silicon hardware. &

    Should write a hadamard decoder, and with that probably some string distance
    utils.

    \\ 2018-5-31 &

    I started work on writing string distance utils, with a bit of research on
    Hamming vs Levenshtein. &

    It's apparently actually quite difficult to do efficiently. I will keep
    working but maybe also think of some other options.

    \\ 2018-6-1 &

    I realised that I would probably need a lot more explanation of what the
    code all does so I started writing documentation and comments in earnest. I
    also added a little preamble to my dissertation talking about comment
    conventions. &

    Had some trouble deciding what kind of format/tone/expertise to use in my
    comments. &

    I should probably keep doing this and think about accessibility to examiners
    elsewhere too.

    \\ 2018-6-2 &

    I wrote some more documentation, and also added some definitions for
    frequently used computing jargon to the preamble of my document. I think
    this will be good for the readability of the whole affair. &

    I wasn't entirely sure how best to format definitions, but I found a good
    solution online. &

    Make sure all of my paperwork is up to date and on OneDrive &

    \\ 2018-6-4 &

    I got my paperwork on OneDrive\ldots &

    &

    I should neaten up my mindmap and notes

    \\

    \bottomrule
    \end{longtable}
    }

    \end{center}

\end{document}
