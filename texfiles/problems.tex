\documentclass{article}
\title{Development of skills for my EP}
\author{Izaak van Dongen}

\usepackage{fullpage}

\usepackage{hologo}

\usepackage[hidelinks]{hyperref}

\usepackage{amsmath}
\usepackage{amsfonts}
\usepackage{commath}

\usepackage{longtable}
\usepackage{booktabs}

\usepackage[parfill]{parskip}

\usepackage[utf8]{inputenc}
\usepackage[T1]{fontenc}

\usepackage[square]{natbib}

\usepackage{graphicx}

\graphicspath{ {images/} }

\usepackage{listings}
\usepackage{color}

\definecolor{codegreen}{rgb}{ 0,0.6,0}
\definecolor{codegray}{rgb}{0.5,0.5,0.5}
\definecolor{codepurple}{rgb}{0.58,0,0.82}
\definecolor{backcolour}{rgb}{0.95,0.95,0.92}
\lstdefinestyle{mystyle}{
    backgroundcolor=\color{backcolour},
    commentstyle=\color{codegreen},
    keywordstyle=\color{magenta},
    numberstyle=\tiny\color{codegray},
    stringstyle=\color{codepurple},
    basicstyle=\footnotesize,
    breaklines=true,
    captionpos=b,
    keepspaces=true,
    numbers=left,
    numbersep=5pt,
    showspaces=false,
    showstringspaces=false,
    showtabs=false,
    tabsize=2
}

\lstset{style=mystyle}

\lstset{
    literate={~} {$\sim$}{1}
}

\begin{document}
    \maketitle

    \section{Problems and how they were overcome}

    \begin{center}
    {
    \renewcommand{\arraystretch}{2.0}
    \begin{longtable}{p{0.2\linewidth} p{0.7\linewidth}} \toprule

    Problem & What I did to overcome it \\ \midrule

    Little to no knowledge of the basic usage of \LaTeX &

    In order to develop my basic knowledge of \LaTeX, I mainly used the online
    resources \url{https://www.latex-project.org/about/} and
    \url{https://www.sharelatex.com/}

%TODO add more about my preamble, find some URLS and add more detail

    \\

    I wanted to customise my document to take up more of the page &

    I installed and loaded savetrees

    \\

    I wanted to include code I'd written in my \LaTeX document &

    I found out about the listings package and found a suitable configuration
    online, that I added to somewhat with other sources. I also figured out
    that I could have \LaTeX directly read from my source files rather than
    having to include them in my .tex file.

    \\

    I had to create a bibliography, using \LaTeX, or, as I now know,
    \hologo{BibTeX}. &

    I did some research on how bibliographies worked in \LaTeX. I had already
    heard of \hologo{BibTeX} citations, and in fact most of the sources I'd
    used up to this point had provided an option to ``export \hologo{BibTeX}
    citation''. Because of this, I could narrow my search terms somewhat, and
    soon found these pages: \url{http://www.bibtex.org/Using/} and 
    \url{https://www.sharelatex.com/learn/Bibliography_management_with_bibtex}.
    With these, I'd soon compiled my own .bib file with extensive records of
    all of my sources.

    \\

%TODO find URLS

    Creating specifically a \textbf{Harvard-referenced} bibliography,
    customised to my liking. &

    As this seemed to be a requirement of the EP qualification, I needed to
    modify my \hologo{BibTeX} workflow to the point where everything was
    Harvard-referenced with author-year citations. I found that this was
    apparently quite difficult to do with vanilla \hologo{BibTeX} so I decided
    to use natbib with the agsm style\ldots

    \\

%TODO replicate and insert error, check fullpage

    Loading the natbib package caused \LaTeX to crash while compiling the
    output PDF, with a seemingly gibberish error &

    After much Googling to no avail, I applied an approach I'd picked up from
    programming PostScript (another language that compiled to PDF for my
    purposes). When I first started programming PostScript, I didn't use an
    interpreter but rather Preview, so I didn't have access to error messages.
    The craft to finding bugs was to section off parts of the source with
    comments and then slowly release them again until you found the bug. By
    doing this I found that loading savetrees was causing the crash. I didn't
    end up finding out what the actual conflict was being caused by, but
    instead used the alternative fullpage package.

    \\

%TODO fact check

    Creating nice-looking, effective tables in \LaTeX. &

    For this project I had to create a number of tables, for various documents
    such as my source evaluation, this document, my diary and at some point I
    played around with some tables in my dissertation. These tables generally
    needed to be very long, which standard \LaTeX ``tabular'' mode didn't
    support.
    I found out about the longtabs package, together with generic tabulation
    and also had to increase my arrayfill to make a spacing between my table
    cells that I found satisfying.

    \\

    \bottomrule
    \label{tab:sourceeval}
    \end{longtable}
    }

    \ref{tab:sourceeval}
    \end{center}

\end{document}
