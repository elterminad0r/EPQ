\documentclass{article}
\title{Development of skills for my EP}
\author{Izaak van Dongen}

% fonts
\usepackage[p,osf]{cochineal}
\usepackage[scale=.95,type1]{cabin}
\usepackage[cochineal,bigdelims,cmintegrals,vvarbb]{newtxmath}
% fixed width font with 80 chars per listing line
\usepackage[scaled=.94]{newtxtt}
\usepackage[cal=boondoxo]{mathalfa}

% make the document take up more of the page
\usepackage[margin=1in,headheight=13.6pt]{geometry}

\usepackage{hologo}

\usepackage[hidelinks]{hyperref}

\usepackage{longtable}
\usepackage{booktabs}

\usepackage[parfill]{parskip}

\usepackage[utf8]{inputenc}
\usepackage[T1]{fontenc}

\usepackage[square]{natbib}

\begin{document}
    \maketitle

    \section{Problems and how they were overcome}

    \begin{center}
    {
    \renewcommand{\arraystretch}{1.5}
    \begin{longtable}{p{0.2\linewidth} p{0.7\linewidth}} \toprule

    Problem & What I did to overcome it \\ \midrule

    Little to no knowledge of the basic usage of \LaTeX &

    In order to develop my basic knowledge of \LaTeX, I mainly used the online
    resources \url{https://www.latex-project.org/about/} and
    \url{https://www.sharelatex.com/}

%TODO add more about my preamble, find some URLS and add more detail

    \\

    I wanted to customise my document to take up more of the page &

    By default, \LaTeX\ uses really large margins, to the point where I found it
    kind of ugly, and most of all a waste of paper. I found that this could be
    solved with a package called `savetrees'. Another option I found was to set
    the margin myself by loading the geometry package.

    \\

    I wanted to include code I'd written in my \LaTeX document &

    I found out about the listings package and found a suitable configuration
    online, that I added to somewhat with other sources. I also figured out
    that I could have \LaTeX directly read from my source files rather than
    having to include them in my .tex file.

    \\

    I had to create a bibliography, using \LaTeX, or, as I now know,
    \hologo{BibTeX}. &

    I did some research on how bibliographies worked in \LaTeX. I had already
    heard of \hologo{BibTeX} citations, and in fact most of the sources I'd
    used up to this point had provided an option to ``export \hologo{BibTeX}
    citation''. Because of this, I could narrow my search terms somewhat, and
    soon found these pages: \url{http://www.bibtex.org/Using/} and
    \url{https://www.sharelatex.com/learn/Bibliography_management_with_bibtex}.
    With these, I'd soon compiled my own .bib file with extensive records of
    all of my sources.

    \\

%TODO find URLS

    Creating specifically a \textbf{Harvard-referenced} bibliography,
    customised to my liking. &

    As this seemed to be a requirement of the EP qualification, I needed to
    modify my \hologo{BibTeX} workflow to the point where everything was
    Harvard-referenced with author-year citations. I found that this was
    apparently quite difficult to do with vanilla \hologo{BibTeX} so I decided
    to use natbib with the agsm style\ldots

    \\

%TODO replicate and insert error, check fullpage

    Loading the natbib package caused \LaTeX\ to crash while compiling the
    output PDF, with a seemingly gibberish error &

    After much Googling to no avail, I applied an approach I'd picked up from
    programming PostScript (another language that compiled to PDF for my
    purposes). When I first started programming PostScript, I didn't use an
    interpreter but rather Preview, so I didn't have access to error messages.
    The craft to finding bugs was to section off parts of the source with
    comments and then slowly release them again until you found the bug. By
    doing this I found that loading savetrees was causing the crash. I didn't
    end up finding out what the actual conflict was being caused by, but
    instead used the alternative fullpage package.

    \\

%TODO fact check

    Creating nice-looking, effective tables in \LaTeX. &

    For this project I had to create a number of tables, for various documents
    such as my source evaluation, this document, my diary and at some point I
    played around with some tables in my dissertation. These tables generally
    needed to be very long, which standard \LaTeX ``tabular'' mode didn't
    support.
    I found out about the longtabs package, together with generic tabulation
    and also had to increase my arrayfill to make a spacing between my table
    cells that I found satisfying.

    \\

    Creating a unique-feeling document with \LaTeX &

    \LaTeX\ has many virtues, but it has a particularly distinctive look, at
    first, which is mainly down to the Computer Modern font. In order to set my
    project apart a bit, I spent some time experimenting with different fonts
    and how to load them properly in \LaTeX. I ended up with what you're seeing
    now, and am overall quite pleased. I also added many other package for small
    visual tweaks. See \texttt{preamble\_compr.pdf} for before/after comparison
    of my \LaTeX\ setup. \\

    Effectively communicating the maths and computing related parts of my
    project. &

    These are subjects where I have relatively more expertise than a lot of
    people. Because of this, I find it easy to gloss over things that really are
    quite important to explain, and just in general to take it all a bit too
    fast. After discussing with my EP tutor, who said to think of it like
    explaining things to my grandma, I asked my grandma for what she would find
    helpful. Because of this, I decided to add more visual explanations for
    things. You can see this in the assortment of figures in my dissertation.

    I also realised that I needed to be more obvious about explaining what my
    code does, seeing as it constitutes such a large portion of my project.
    Because of this, I started extensively commenting all of my source code, not
    just giving a general indication of what the program does, but really trying
    to give as much of an explanation of as much of the code as I can.

    \\

    Bugs in the code! &

    Frequently thoughout this project I would have problems running various bits
    of code. This was especially stressfull as my dissertation is comprised of
    bits of code in itself. When this happened I would have to adapt, and use
    various techniques to try and fix things. After searching for any error
    messages or stack dumps generated by buggy code, I would just start removing
    bits of code until it worked again. This would tell me which bits were
    broken, and from there it's generally a lot easier.

    This general problem was also helped by starting to more rigorously test the
    Python programs I wrote, with the unit testing idiom. By ensuring that each
    component works as it should, it's a lot easier to be confident in your
    program. It also makes it a lot easier to track down which component is
    causing the problems if it does break again.

    \\

    \bottomrule
    \end{longtable}
    }

    \end{center}

\end{document}
