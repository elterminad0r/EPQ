\documentclass{article}
\title{The application of noisy-channel coding techniques to DNA barcoding (Early structure/ideas)}
\author{Izaak van Dongen}

\usepackage{fullpage}

\usepackage[hidelinks]{hyperref}

\usepackage{amsmath}
\usepackage{amsfonts}
\usepackage{commath}

\usepackage{courier}

\usepackage[nottoc]{tocbibind}

\usepackage[parfill]{parskip}

\usepackage[utf8]{inputenc}
\usepackage[T1]{fontenc}

\usepackage[square]{natbib}

\usepackage{graphicx}

\graphicspath{ {images/} }

\usepackage{listings}
\usepackage{color}

\definecolor{codegreen}{rgb}{ 0,0.6,0}
\definecolor{codegray}{rgb}{0.5,0.5,0.5}
\definecolor{codepurple}{rgb}{0.58,0,0.82}
\definecolor{backcolour}{rgb}{0.95,0.95,0.92}

\lstdefinestyle{mystyle}{
    backgroundcolor=\color{backcolour},
    commentstyle=\color{codegreen},
    keywordstyle=\color{magenta},
    numberstyle=\tiny\color{codegray},
    stringstyle=\color{codepurple},
    basicstyle=\footnotesize,
    breaklines=true,
    postbreak=\mbox{\textcolor{red}{$\hookrightarrow$}\space},
    captionpos=b,
    keepspaces=true,
    numbers=left,
    numbersep=5pt,
    showspaces=false,
    showstringspaces=false,
    showtabs=false,
    tabsize=2
}

\lstset{style=mystyle}

\lstset{
    literate={~} {$\sim$}{1}
}

\begin{document}
    \maketitle
    \tableofcontents
    \lstlistoflistings

    \section{Introduction}
    The premise of this project is to investigate the different types of
    error-correcting codes, and how these might be applied to DNA barcoding.
    The challenge in this comes from the fact that most error-correcting codes
    are designed in base-2 (binary) whereas DNA strings are fundamentally
    base-4 (quaternary). The applicability of this project is that in
    oligonucleotide synthesis, some samples may need to be identified later on
    using a subsection of the sample (a barcode). These could just be linearly
    assigned codes, but this would leave them very susceptible to mutation.

    Here is an example: say that we're given a barcode of length four, to
    encode two different samples. If we worked methodically up from the bottom
    (using the ordering ACGT - orderings will be discussed further later on) we
    might end up with the codes AAAA and AAAC. However, either string would
    only require a single mutation (where we say a mutation is the changing of
    a single base) to become identical to the other one. Therefore, in this
    case, it would clearly be far more optimal to make a choice like, for
    example, AAAA and CCCC.

    There have been a few assumptions and glossed over definitions here:

    \begin{itemize}
        \item What constitutes a mutation?
        \item What is the best way to represent DNA mathematically?
    \end{itemize}

    There are also a number of parameters to the problem, and as they change
    the problem becomes very much nontrivial:

    \begin{itemize}
    \item What if the barcode size changes?
    \item What if we want more codes than two?
    \item What if rather than number of codes and barcode size, the parameters
          are set to barcode size and maximum number of mutations that can
          occur?
    \end{itemize}
    
    All of these will be further explored in this dissertation.

    \section{The Hamming distance}

    The Hamming distance is a measure of ``string distance''. String
    distance is a way to define how different two string are.
    Coding-theoretically, this can be used to quantify the amount that a
    string has been changed by transmission (or an oligonucleotide has been
    mutated).

    The Hamming distance between any two equally long strings $S$ and $R$
    is given by the number of characters at identical position that differ.
    For example, the distances 
    \begin{align*}
    d(S, R) &= 1\\
    d(S, T) &= 2\\
    \text{where}\\
    S &= \texttt{abc\textcolor{red}{d}e}\\
    R &= \texttt{abc\textcolor{blue}{f}e}\\
    T &= \texttt{a\textcolor{codegreen}{x}c\textcolor{codegreen}{z}e}
    \end{align*}
    Note that for any $S$, $d(S, S) = 0$. This means that there is no
    ``distance'' from a string to itself.

    In terms of DNA, the Hamming distance can be used to determine the
    number of bases that have mutated.

%TODO citation needed

    \section{Parity codes}

    A simple but inefficient parity encoding scheme is a column/row wise
    encoding. Take the slightly contrived data string ``0100000101010100''.
    This is very tangentially related to DNA - it's the 8-bit ASCII
    representation of the string ``AT'', generated by the Python:
    \lstinline[language=Python]|"".join("0" + bin(ord(c))[2:] for c in "AT")|

%TODO cite ASCII

    Anyway, the string is then split into a square like so:

    \begin{center}
    \begin{math}
    \begin{matrix}
        0 & 1 & 0 & 0 \\ 
        0 & 0 & 0 & 1 \\
        0 & 1 & 0 & 1 \\
        0 & 1 & 0 & 0
    \end{matrix}
    \end{math}
    \end{center}

    An extra row and column, including an extra corner piece is appended like
    so:

    \begin{center}
    \begin{tabular}{c c c c | c}
        0 & 1 & 0 & 0 & 1 \\ 
        0 & 0 & 0 & 1 & 1 \\
        0 & 1 & 0 & 1 & 0 \\
        0 & 1 & 0 & 0 & 1 \\ \hline
        0 & 1 & 0 & 0 & 1
   \end{tabular}
   \end{center}

   Each of the extra bits documents the pairty of its row.

%TODO show optimum row-col distribution

    \section{The Hamming code}

    The Hamming code is a binary encoding scheme that uses parity to
    detect, and correct errors. The insertion of ``parity bits'' is a
    common practice in basic encoding. Parity refers to the ``oddness'' or
    ``evenness'' of some data. Commonly, this is determined by the sum of
    the data modulo 2. For example, ``00101'' results in a parity bit of 0,
    because the sum of all the bits is 2, which has a remainder of 0 when
    divided by 2 (is equal to 0 mod 2).

    \section{Implementing the Hamming code}

    The script implementing a simple binary Hamming code is as follows:

\lstinputlisting[language=Python, caption=Binary Hamming code in Python]
{../src/binary_hamming.py}

    This code is accompanied by the following testing scheme:

\lstinputlisting[language=Python, caption=binary\_hamming unit tests]
{../src/test_binary_hamming.py}

    \section{The Hadamard code}

    \section{Source}

    All code and source \TeX/\LaTeX{} files can be found at
    \url{https://github.com/elterminad0r/EPQ}.

\nocite{*}

\bibliographystyle{agsm}
\bibliography{sources}

\end{document}
