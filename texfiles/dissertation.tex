\documentclass{article}
\title{The application of noisy-channel coding techniques to DNA barcoding (Early structure/ideas)}
\author{Izaak van Dongen}

\usepackage{fullpage}

\usepackage{amsmath}
\usepackage{amsfonts}
\usepackage{commath}

\usepackage[nottoc]{tocbibind}

\usepackage[parfill]{parskip}

\usepackage[utf8]{inputenc}
\usepackage[T1]{fontenc}

\usepackage[square]{natbib}

\usepackage{graphicx}

\graphicspath{ {images/} }

\usepackage{listings}
\usepackage{color}

\definecolor{codegreen}{rgb}{ 0,0.6,0}
\definecolor{codegray}{rgb}{0.5,0.5,0.5}
\definecolor{codepurple}{rgb}{0.58,0,0.82}
\definecolor{backcolour}{rgb}{0.95,0.95,0.92}

\lstdefinestyle{mystyle}{
    backgroundcolor=\color{backcolour},
    commentstyle=\color{codegreen},
    keywordstyle=\color{magenta},
    numberstyle=\tiny\color{codegray},
    stringstyle=\color{codepurple},
    basicstyle=\footnotesize,
    breaklines=true,
    postbreak=\mbox{\textcolor{red}{$\hookrightarrow$}\space},
    captionpos=b,
    keepspaces=true,
    numbers=left,
    numbersep=5pt,
    showspaces=false,
    showstringspaces=false,
    showtabs=false,
    tabsize=2
}

\lstset{style=mystyle}

\lstset{
    literate={~} {$\sim$}{1}
}

\begin{document}
    \maketitle
    \tableofcontents
    \lstlistoflistings

    \section{Introduction}
    The premise of this project is to investigate the different types of
    error-correcting codes, and how these might be applied to DNA barcoding.
    The challenge in this comes from the fact that most error-correcting codes
    are designed in base-2 (binary) whereas DNA strings are fundamentally
    base-4 (quaternary). The applicability of this project is that in
    oligonucleotide synthesis, some samples may need to be idnetified later on
    using a subseciton of the sample (a barcode). These could just be linearly
    assigned codes, but this would leave them very susceptable to mutation.

    Here is an example: say that we're given a barcode of length four, to
    encode two different samples. If we worked methodically up from the bottom
    (using the ordering ACGT - orderings will be discussed further later on) we
    might end up with the codes AAAA and AAAC. However, either string would
    only require a single mutation (where we say a mutation is the changing of
    a single base) to become identical to the other one. Therefore, in this
    case, it would clearly be far more optimal to make a choice like, for
    example, AAAA and CCCC.

    There have been a few assumptions and glossed over definitions here:

    \begin{itemize}
        \item What constitutes a mutation?
        \item What is the best way to represent DNA mathematically?
    \end{itemize}

    There are also a number of parameters to the problem, and as they change
    the problem becomes very much nontrivial:

    \begin{itemize}
    \item What if the barcode size changes?
    \item What if we want more codes than two?
    \item What if rather than number of codes and barcode size, the parameters
          are set to barcode size and maximum number of mutations that can
          occur?
    \end{itemize}
    
    All of these will be further explored in this dissertation.

    \section{The Hamming distance}

    \section{Implementing the Hamming code}

    The script implementing a simple binary Hamming code is as follows:

\lstinputlisting[language=Python, caption=Binary Hamming code in Python]{../src/binary_hamming.py}

    This code is accompanied by the following testing scheme:

\lstinputlisting[language=Python, caption=binary\_hamming unit tests]{../src/test_binary_hamming.py}

    \section{Source}

    All code and source \TeX/\LaTeX files can be found at
    \verb|https://github.com/elterminad0r/EPQ|.

\nocite{*}

\bibliographystyle{agsm}
\bibliography{sources}

\end{document}
