\documentclass{article}
\title{Source evaluation}
\author{Izaak van Dongen}

\usepackage{savetrees}

\usepackage{amsmath}
\usepackage{amsfonts}
\usepackage{commath}

\usepackage[parfill]{parskip}

\usepackage[utf8]{inputenc}
\usepackage[T1]{fontenc}

\usepackage{graphicx}

\graphicspath{ {images/} }

\usepackage{listings}
\usepackage{color}

\definecolor{codegreen}{rgb}{ 0,0.6,0}
\definecolor{codegray}{rgb}{0.5,0.5,0.5}
\definecolor{codepurple}{rgb}{0.58,0,0.82}
\definecolor{backcolour}{rgb}{0.95,0.95,0.92}
\lstdefinestyle{mystyle}{
    backgroundcolor=\color{backcolour},
    commentstyle=\color{codegreen},
    keywordstyle=\color{magenta},
    numberstyle=\tiny\color{codegray},
    stringstyle=\color{codepurple},
    basicstyle=\footnotesize,
    breaklines=true,
    captionpos=b,
    keepspaces=true,
    numbers=left,
    numbersep=5pt,
    showspaces=false,
    showstringspaces=false,
    showtabs=false,
    tabsize=2
}

\lstset{style=mystyle}

\lstset{
    literate={~} {$\sim$}{1}
}

\begin{document}
    \maketitle
    \tableofcontents

    \section{Source evaluation}
    \subsection{Error detecting and error correcting codes \cite{Hamming}}

    This is a paper by R. W. Hamming, who contributed much to modern
    error-correcting codes. One of the main encoding types used in my project
    is even named the Hamming code. It has formed much of the basis of modern
    communication theory and can certainly be trusted.

    \subsection{A mathematical theory of communication \cite{Shannon}}

    This paper, together with \cite{Hamming} are generally considered to be the
    seminal works on coding theory. This lays much of the groundwork for
    communication theory and gives a more general definition of the Hamming
    Code

    \subsection{Generalized dna barcode design based on hamming codes \cite{HammingBarcodes}}

    This article is very much relevant to what my project is about. It doesn't
    seem to be very clear though, and it uses a seemingly inoptimal form of
    parity. However it provides helpful insight into what actual researchers in
    the field are doing and have done with these ideas.

    \subsection{Introduction to coding theory \cite{CodeIntro}}

    This is not a great academic source, but very helpful for reaching more of
    a `layman's' kind of understanding of coding theory.

\bibliography{sources}{}
\bibliographystyle{unsrt}

\end{document}
