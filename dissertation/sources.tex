\documentclass{article}
\title{Source evaluation}
\author{Izaak van Dongen}

\usepackage{fullpage}

\usepackage{amsmath}
\usepackage{amsfonts}
\usepackage{commath}

\usepackage[parfill]{parskip}

\usepackage[utf8]{inputenc}
\usepackage[T1]{fontenc}

\usepackage{natbib}

\usepackage{graphicx}

\graphicspath{ {images/} }

\usepackage{listings}
\usepackage{color}

\definecolor{codegreen}{rgb}{ 0,0.6,0}
\definecolor{codegray}{rgb}{0.5,0.5,0.5}
\definecolor{codepurple}{rgb}{0.58,0,0.82}
\definecolor{backcolour}{rgb}{0.95,0.95,0.92}
\lstdefinestyle{mystyle}{
    backgroundcolor=\color{backcolour},
    commentstyle=\color{codegreen},
    keywordstyle=\color{magenta},
    numberstyle=\tiny\color{codegray},
    stringstyle=\color{codepurple},
    basicstyle=\footnotesize,
    breaklines=true,
    captionpos=b,
    keepspaces=true,
    numbers=left,
    numbersep=5pt,
    showspaces=false,
    showstringspaces=false,
    showtabs=false,
    tabsize=2
}

\lstset{style=mystyle}

\lstset{
    literate={~} {$\sim$}{1}
}

\begin{document}
    \maketitle

    \section{Error detecting and error correcting codes \citep*{Codes1950Hamming}}

    This is a paper by R. W. Hamming, who contributed much to modern
    error-correcting codes. One of the main encoding types used in my project
    is even named the Hamming code. It has formed much of the basis of modern
    communication theory and can certainly be trusted.

    \section{A mathematical theory of communication \citep*{Communication1948Shannon}}

    This paper, together with \citep*{Codes1950Hamming} are generally considered
    to be the seminal works on coding theory. This lays much of the groundwork
    for communication theory and gives a more general definition of the Hamming
    Code

    \section{Generalized dna barcode design based on hamming codes \citep*{HammingBarcodes2012BystrykhLeonid}}

    This article is very much relevant to what my project is about. It doesn't
    seem to be very clear though, and it uses a seemingly inoptimal form of
    parity. However it provides helpful insight into what actual researchers in
    the field are doing and have done with these ideas.

    \section{Introduction to coding theory \citep*{CodeIntro2010Guruswami}}

    This is not a great academic source, but very helpful for reaching more of
    a `layman's' kind of understanding of coding theory.

    \section{Polynomial codes: an optimal design for high-dimensional coded matrix multiplication \citep*{PolynomialCodes2017MaddahAvestimehr}}

    This source is very technically detailed, which isn't necessarily a bad
    thing but makes it pretty dense. Potentially very useful for a complicated
    understanding of polynomial codes, although I'm not sure if I'll use
    polynomial codes.

    \section{Families of hadamard z2z4q8-codes \citep*{HadamardZ2Z2012RioRifa}}

    This source turned out not to be very useful as it only relates to a highly
    specific class of Hadamard code. Other simpler tutorials on the internet
    are much more useful.

    \section{Hadamard matrices and their applications \citep*{HadamardMatrices1978HedayatWallis}}

    This is a far more appropriate paper that gives a more general overview of
    what a hadamard matrix is and can be used for. Very useful overall.

    \section{The search for hadamard matrices \citep*{HadamardSearch1963GolombBaumert}}

    \section{Hadamard matrices and their designs: A coding-theoretic approach \citep*{HadamardCodingTheoretic1992AssmusKey}}

    \section{Hadamard designs \citep*{HadamardDesigns1972Spence}}

    \section{Lifted polynomials over ${F}_{16}$ and their applications to dna codes \citep*{PolynomialDNA2013OztasSiap}}

    \section{Codes, not ciphers \citep*{CodesCiphers1010Baylis}}

    \section{Error correcting codes: Practical origins and mathematical implications \citep*{CodesOrigins1978Pless}}

\bibliographystyle{agsm}
\bibliography{sources}

\end{document}
